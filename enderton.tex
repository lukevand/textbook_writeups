\documentclass[12pt]{article}

\author{Luke van Duuren}
\title{Herbert Enderton's\\ Elements of Set Theory}

\usepackage{amssymb}
\usepackage{amsthm}
\usepackage{mathrsfs} % for \mathscr
\usepackage{mathtools}
\usepackage[shortlabels]{enumitem}
\usepackage{multicol}

\setlength{\parindent}{0pt}

\theoremstyle{plain}
\newtheorem{thm}{Theorem}[section]
\newtheorem*{exthm}{Theorem}

\theoremstyle{remark}
\newtheorem*{eg}{Example}

\theoremstyle{definition}
\newtheorem{axiom}{Axiom}[section]
\newtheorem{defn}{Definition}[section]

\theoremstyle{remark}
\newtheorem*{remark}{Remark}

\newcommand{\powerset}{\mathscr{P}\,}

\newcommand{\thmproof}[3]{%
  \begin{thm}[#1]
    #2
  \end{thm}
  \begin{proof}
    #3
  \end{proof}
}

\providecommand\st{}

\newcommand\SetSymbol[1][]{%
  \nonscript\:#1\vert%
  \allowbreak%
  \nonscript\:
\mathopen{}}

\DeclarePairedDelimiter{\ang}{\langle}{\rangle}

\DeclarePairedDelimiterX\Set[1]\{\}{%
  \renewcommand\st{\SetSymbol[\delimsize]}
  #1
}

\DeclarePairedDelimiter\aSet\{\}
\renewcommand{\iff}{\Leftrightarrow}

\DeclareMathOperator{\dom}{dom}
\DeclareMathOperator{\ran}{ran}
\DeclareMathOperator{\fld}{fld}

\begin{document}
\maketitle

\section{Introduction and Motivation}
There are my notes to the book. It also contains extra theorems to ``obvious'' facts.

I have started reading Enderton a few times previously, like 2020, when I started writing this!

\section{Axioms}
\begin{axiom}[Extensionality Axiom]\label{extensionality}
  If two sets have exactly the same members, then they are equal:
  \[
    \forall A\ \forall B \left(\forall x(x \in A \Leftrightarrow x \in B) \Rightarrow A = B\right).
  \]
\end{axiom}

\begin{axiom}[Empty Set Axiom] There is a set having no members:
  \[
    \exists B \ \forall x \ x \not\in B.
  \]
\end{axiom}

This can be rewritten as
\[
  \exists B \ \forall x (x \in B\ \iff\ x \neq x),
\]
\begin{axiom}[Pairing Axiom] For any sets $u$ and $v$, there is a set having members of just $u$ and $v$:
  \[
    \forall u\ \forall v\ \exists B\ \forall x(x \in B \Leftrightarrow x = u \text{ or } x = v).
  \]
\end{axiom}

\begin{axiom}[Power Set Axiom] For any set $a$, there is a set whose members are exactly the subsets of $a$:
  \[
    \forall a\ \exists B\ \forall x(x \in B \Leftrightarrow x \subseteq a).
  \]
\end{axiom}

\begin{axiom}[Subset Axioms]
  For each formula $\alpha$ not containing $B$, the following is an axiom:
  \[
    \forall t_1\ \cdots\ \forall t_k\ \forall c\ \exists B\ \forall x(x \in B \Leftrightarrow x \in c \text{ and } \alpha).
  \]
\end{axiom}

\begin{axiom}[Union Axiom]
  For any set $A$ there exists a set $B$ whose elements are exactly the members of the members of $A$:
  \[
    \forall x(x \in B \Leftrightarrow (\exists b \in A) x \in b).
  \]
\end{axiom}

\begin{axiom}[Axiom of Choice (first form)]
  For any relation $R$ there is a function $H \subseteq R$ with $H = \dom{R}$.
\end{axiom}

\section{Definitions}
\begin{defn}
  The concepts of ``set'' and ``member'' are \textit{primitive notions} which remain undefined.
\end{defn}

\begin{defn}
  \textit{Logical consequences} or \textit{theorems} are derived sentences of the list of axioms. A sentence $\sigma$ is a \textit{logical consequence} of the axioms if any assignment of meaning to the undefined notions of set and member making the axioms true also make $\sigma$ true.
\end{defn}

\begin{defn}\label{emptyset}
  $\varnothing$ is the set having no members.
\end{defn}

\begin{defn}
  For any sets $u$ and $v$, the \textit{pair set} $\{u,\ v\}$ is the set whose members are only $u$ and $v$.
\end{defn}

\begin{defn}
  For any set $a$ and $b$, the \textit{union} $a \cup b$ is the set whose members are those sets belonging to either $a$ or $b$.
\end{defn}

\begin{defn}
  For any set $a$, the \textit{power set} $\powerset a$ is the set whose members are exactly the subsets of $a$.
\end{defn}

\begin{defn}
  For any $x$, the \textit{singleton} $\{x\}$ is the set formed by $\{x,\ x\}$.
  \begin{remark}
    We can form the set $\{x,\ x\}$ by the Pairing Axiom.
  \end{remark}
\end{defn}

\begin{defn}
  For any sets $x_1, x_2, x_3$,
  \[
    \{x_1,\ x_2,\ x_3\} := \{x_1,\ x_2\} \cup \{x_3\}.
  \]
\end{defn}

\begin{defn}
  The \textit{union} $\cup A$ of $A$ is the set
  \begin{align*}
    \cup A &= \Set*{x \st \text{$x$ belongs to some member of $A$}}\\
           &= \Set*{x \st (\exists b \in A)\, x \in b}.
  \end{align*}
\end{defn}

\begin{defn}
  For any sets $A$ and $B$, the \textit{relative complement $A-B$ of $B$ in $A$}:
  \[
    A - B = \Set*{x \in A \st x \not\in B}.
  \]
\end{defn}

\begin{defn}
  The \textit{symmetric difference} $A+B$ of sets $A$ and $B$ is $(A - B) \cup (B - A)$.
\end{defn}

\begin{defn}
  The \textit{ordered pair} $\ang{x, y}$ is defined to be $\aSet{\aSet{x}, \aSet{x, y}}$.
\end{defn}

\begin{defn}
  The \textit{Cartesian product} $A \times B$ of $A$ and $B$ is:
  \[
    A \times B = \Set*{\ang{x,y} \st x \in A\ \text{and}\ y \in B}.
  \]
\end{defn}

\begin{defn}
  A \textit{relation} is a set of ordered pairs.
\end{defn}

\begin{defn}
  The \textit{domain} of $R$ (dom $R$), the \textit{range} of $R$ (ran $R$), and the \textit{field} of $R$ (fld $R$) is given by
  \begin{align*}
    x \in \dom{R} &\iff \exists y \ang{x, y} \in R,\\
    x \in \ran{R} &\iff \exists t \ang{t, x} \in R,\\
    \fld{R} &= \dom{R} \cup \ran{R}.
  \end{align*}
\end{defn}

\begin{defn}
  $n$-tuples are defined as follows. For triples, $\ang{x,y,z} = \ang{\ang{x,y},z }$ and so on. The 1-tuple is defined $\ang{x} = x$.
\end{defn}

\begin{defn}
  A \textit{n-ary relation on $A$} is defined to be a set of ordered $n$-tuples with all components in $A$.
\end{defn}

\begin{defn}
  A \textit{function} is a relation $F$ such that for each $x$ in $\dom F$ there is only one $y$ such that $xFy$.
\end{defn}

\begin{defn}
  The \textit{identity relation} on $\omega$ is
  \[
  I_\omega = \Set{\ang{n,n} \st n \in \omega}.
  \]
\end{defn}

\begin{defn}
  For a function $F$ and a point $x$ in $\dom F$, the unique $y$ such that $xFy$ is called the \textit{value of $F$ at $x$}, denoted by $F(x)$.
\end{defn}

\begin{defn}
  A set $R$ is \textit{single-rooted} iff for each $y \in \ran R$ there is only one $x$ such that $xRy$.
\end{defn}

\begin{defn}
  The \textit{composition} of $F$ and $G$ is the set
  \[
    F \circ G = \Set{\ang{u, v} \st \exists t (uGt \text{ and } tFv)}.
  \]
\end{defn}

\begin{defn}
  The \textit{restriction} of $F$ to $A$ is the set
  \[
    F \restriction A = \Set*{\ang{u,v} \st uFv \text{ and } u \in A}
  \]
\end{defn}

\begin{defn}
  The \textit{image of $A$ under $F$} is the set
  \begin{align*}
    F [A] &= \ran{(F \restriction A)}\\
          &= \Set{v \st (\exists u \in A)\ uFv}
  \end{align*}
\end{defn}

\begin{defn}
  The set of all functions from $A$ to $B$ is
  \[
    ^A B = \Set*{F \st \text{F is a function from $A$ into $B$}}
  \]
  read ``B-pre-A''.
\end{defn}

\begin{defn}
  $R$ is an \textit{equivalence relation on $A$} iff $R$ is a binary relation on $A$ that is reflexive on $A$, symmetric, and transitive.
\end{defn}

\begin{defn}
  The set $[x]_R$ is defined by
  \begin{equation*}
    [x]_R = \Set*{t \st xRt}
  \end{equation*}
\end{defn}

\begin{defn}
  For any set $a$, it's \textit{successor} $a^+$ is defined by
  \begin{equation*}
    a^+ = a \cup \{a\}.
  \end{equation*}
\end{defn}

\begin{defn}
  A set $A$ is \textit{inductive} iff $\varnothing \in A$ and it is closed under successor, ie for every $a \in A$, $a^+ \in A$.
\end{defn}

\begin{defn}
  A \textit{natural number} is a set that belongs to every inductive set.
\end{defn}

\section{Theorems}
\thmproof{}{There cannot be two different sets, each of which has no members.}
{Let $A$ and $B$ be sets, each of which has no members. Then exactly the same things belong to $A$ as to $B$. By the Extensionality Axiom, $A = B$.
}

\begin{thm}\label{identity}
  Let $A$ be a set. Then $A = A$.
\end{thm}
\begin{proof}
  Since for all $x$, $x \in A$ if and only if $x \in A$, by the Extensionality Axiom, we have $A = A$.
\end{proof}

\begin{thm}
  $\varnothing = \Set*{x \st x \neq x}$.
\end{thm}
\begin{proof}
  Let $y$ be a set, $\varnothing' = \{x\ |\ x \neq x\}$ and suppose $y \in \varnothing'$. Then $y \neq y$. But by Theorem~\ref{identity}, $y = y$, a contradiction. Therefore, the premise $y \in \varnothing'$ is false. Hence, $\varnothing'$ has no members, by definition, we have $\varnothing' = \varnothing$.
\end{proof}

\begin{thm}\label{multi}
  Let $x$ be a set. Then $\{x\} = \{x,\ x\}$.
\end{thm}
\begin{proof}
  This follows from the Extensionality Axiom, since for any set $y$, $y \in \{x\} \Leftrightarrow y \in \{x,\ x\}$.
\end{proof}

\begin{thm}
  Let $x$ be a set. Then the set $\{x\}$ exists.
\end{thm}
\begin{proof}
  By the Pairing Axiom, there exists a set having members $x$, or $\{x,\ x\}$. By Theorem~\ref{multi}, $\{x,\ x\} = \{x\}$.
\end{proof}

\begin{thm}
  Let $x$ be a set. Then $\cup x = x$.
\end{thm}
\begin{proof}
  Follows directly from the Union Axiom.
\end{proof}

\thmproof{}{$\cup{\{a\}} = a$.}
{We have
  \begin{align*}
    \cup{\{a\}} &= \{x\ |\ \text{$x$ belongs to some member of $\{a\}$}\}\\
                &= \{x\ |\ x \in a\}\\
                &= a.
  \end{align*}
}

\thmproof{}{$\cup \varnothing = \varnothing$.}
{We have $\cup \varnothing = \{x\ |\ \text{$x$ belongs to some member of $\varnothing$}\}$. But by definition, $\varnothing$ has no members, so there is no $x$ that can be a member of $\varnothing$. Hence, $\cup \varnothing = \varnothing$.
}

\thmproof{}{Whenever $A \subseteq B$, then $\cap B \subseteq \cap A$.}
{Suppose $B$ is non-empty. Let $x \in \cap B$. Then for all $b \in B$, $x \in b$ TODO
}

\thmproof{}{$\powerset \varnothing = \{\varnothing\}$}
{The power set of $\varnothing$ is the set of all subsets of $\varnothing$. Since $\varnothing$ is the only subset of $\varnothing$, $\powerset \varnothing = \{\varnothing \}$.
}

\begin{thm}\label{powersetMember}
  If $b \subseteq A$, then $b \in \powerset A$.
\end{thm}
\begin{proof}
  Since the power set of $A$ is the set of all subsets of $A$, and $b$ is a subset of $A$, $b \in \powerset A$.
\end{proof}

\begin{thm}
  If $A \subseteq B$ then $\cap B \subseteq \cap A$.
\end{thm}
\begin{proof}
  Suppose not. Suppose there exists a $x \in \cap B$ but $x \not\in \cap A$. This means that there is a $a \in A$ such that $x \not\in a$. Since $A \subseteq B$, $a \in B$. But $x$ is a member of \textit{all} members of $B$, a contradiction.
\end{proof}

\begin{thm}
  $\Set{A \cup X \st X \in \mathscr{B}}$ is a set.
\end{thm}
\begin{proof}
  Let $\mathscr{D} = \Set{A \cup X \st X \in \mathscr{B}}$. Let $A \cup X$ be an arbitrary member of $\mathscr{D}$. Since $X \in \mathscr{B}$, $X \subseteq \bigcup \mathscr{B}$, and so $A \cup X \subseteq A \cup \bigcup \mathscr{B}$. Hence $A \cup X \in \powerset\left(A \cup \bigcup  \mathscr{B}\right)$.

  With a subset axiom,
  \[
    \mathscr{D} = \Set*{t \in \powerset\left(A \cup \bigcup  \mathscr{B}\right) \st t = A \cup X \ \text{for some $X$ in $\mathscr{B}$}}.
  \]
\end{proof}

\begin{thm}
  $\Set*{\powerset X \st X \in \mathscr{A}}$ is a set.
\end{thm}
\begin{proof}
  Since $X \in \mathscr{A}$, $X \subseteq \bigcup  \mathscr{A}$. And so $\powerset X \subseteq \powerset \bigcup  \mathscr{A}$. It follows that $\powerset X \in \powerset \powerset \bigcup  \mathscr{A}$. A subset axiom produces this set is
  \[
    \Set*{t \in \powerset \powerset \bigcup  \mathscr{A} \st t = \powerset X\ \text{for some $X$ in $\mathscr{A}$}}.
  \]
\end{proof}

\begin{thmproof}{}
  {For nonempty set $A$, $^{A}\varnothing = \varnothing$.}
  {Let $f \in\, ^{A} \varnothing$ and suppose $\ang{x, y} \in f$, then $x \in A$ and $y \in \varnothing$, which is impossible. Therefore, $^{A} \varnothing = \varnothing$.}
\end{thmproof}

\begin{thmproof}{}
  {For any set $A$, $^{\varnothing} A = \{\varnothing\}$.}
  {Since $\varnothing : \varnothing \rightarrow A$ and $\varnothing$ is the only function with an empty domain.}
\end{thmproof}

\begin{thm}
  The following are equivalent:
  \begin{enumerate}
    \item $x \in a \in A \rightarrow x \in A$
    \item $\bigcup A \subseteq A$
    \item $a \in A \rightarrow a \subseteq A$
    \item $A \subseteq \powerset(A)$.
  \end{enumerate}
\end{thm}
\begin{proof}
  \begin{itemize}
    \item ($1 \rightarrow 2$) Suppose $x \in \bigcup A$. Then there is an $a \in A$, such that $x \in a$. By (1), $x \in A$. Therefore $\bigcup A \subseteq A$.

    \item ($2 \rightarrow 3$) Let $a \in A$. If $a = \varnothing$, then $a \subseteq A$. Suppose then $a \neq \varnothing$, then there is an $x \in a$. Since $x \in a$ implies $x \in \bigcup A$, by (2), $x \in A$. Therefore $a \subseteq A$.

    \item ($3 \rightarrow 4$) Let $a \in A$. By (3), $a \subseteq A$, so $a \in \powerset(A)$.

    \item ($4 \rightarrow 1$) Suppose $x \in a \in A$. By (4), since $a \in A$, $a \in \powerset(A)$, so $x \in a \subseteq A$. Therefore $x \in A$.
  \end{itemize}
\end{proof}

\section{Exercises}
\subsection{Introduction}

\subsubsection{Baby Set Theory}
\begin{exthm}[3]
  If $B \subseteq C$ then $\powerset B \subseteq \powerset C$.
\end{exthm}
\begin{proof}
  Let $A \in \powerset B$. Then $A \subseteq B$. Since $B \subseteq C$, by transitivity of $\subseteq$, we have $A \subseteq C$. So $A \in \powerset C$.
\end{proof}

\begin{eg}
  Define the rank of a set $c$ to be the least $\alpha$ such that $c \subseteq V_\alpha$. Then the rank of $\{\{\varnothing\}\}$ is 2.
\end{eg}

\subsection{Axioms and Operations}

\subsubsection{Arbitrary Unions and Intersections}
\begin{eg}[2]
  $\cup A = \cup B$ does not imply $A = B$.
  Let $A = \{\{1\}, \{2\}\}$ and $B = \{\{1, 2\}\}$.
\end{eg}

\begin{exthm}[3]\label{memberUnion}
  If $b \in A$, then $b \subseteq \cup A$. (Or: Every member of a set $A$ is a subset of $\cup A$.)
\end{exthm}
\begin{proof}
  If $b = \varnothing$, then as the empty set if a subset of every set, we have $\varnothing \subseteq \cup A$.  Suppose $b$ is non-empty and that $a \in b$. Since $b \in A$, $a$ is a member of a member of $A$, so $a \in \cup A$.
\end{proof}
\begin{remark}
  If $A$ is a set, $b \subseteq \cup A$ does not imply $b \in A$. For example, $a = \{1,2\}$ and $B = \aSet{\aSet{1}, \aSet{2}}$.
\end{remark}

\begin{exthm}[4]
  If $A \subseteq B$, then $\cup A \subseteq \cup B$
\end{exthm}
\begin{proof}
  If $A$ is the empty set, this is clearly true. Suppose $A$ is non-empty and let $x \in \cup A$. Then there exists a $a \in A$ such that $x \in a$. But since $A \subseteq B$, $a \in B$, so $x$ is a member of a member of $B$. Therefore $x \in \cup B$.
\end{proof}

\begin{exthm}[Q 6a]
  For any set $A$, $\cup \powerset A = A$
\end{exthm}
\begin{proof}
  \begin{align*}
    \cup \powerset A &= \cup \{X\ |\ \text{$X$ is a subset of $A$} \}\\
                     &= \{x\ |\ \text{$x$ belongs to some subset of $A$}\}\\
                     &= \{x\ |\ \text{$x$ belongs to $A$}\}\\
                     &= A.
  \end{align*}
\end{proof}

\begin{exthm}[Q 6b]
  For any set $A$, $A \subseteq \powerset \cup A$
\end{exthm}
\begin{proof}
  Let $a \in A$. By Exercise~\ref{memberUnion}, $a \subseteq \cup A$ and by Theorem~\ref{powersetMember} we have $a \in \powerset \cup A$.
\end{proof}

\begin{exthm}[6b]
  If $A = \powerset B$ for some set $B$, then $A = \powerset(\cup A)$.
\end{exthm}
\begin{proof}
  \begin{align*}
    \powerset(\cup A) &= \powerset(\cup (\powerset B))\\
                         &= \powerset(B)\\
                         &= A.
  \end{align*}
\end{proof}

\begin{exthm}[7a]
  For any sets $A$ and $B$, $\powerset A \cap \powerset B = \powerset (A \cap B)$.
\end{exthm}
\begin{proof}
  Let $x$ be an arbitrary member of $\powerset A \cap \powerset B$. Then
  \begin{align*}
    x \in \powerset A \text{ and } x \in \powerset B &\Leftrightarrow
    x \subseteq A \text { and } x \subseteq B\\
                                                     &\Leftrightarrow x \subseteq A \cap B\\
                                                     &\iff x \in \powerset (A \cap B)
  \end{align*}
\end{proof}

\begin{exthm}[7b]
  For any sets $A$ and $B$, $\powerset A \cup \powerset B \subseteq \powerset(A \cup B)$.
\end{exthm}
\begin{proof}
  Let $x \in \powerset A \cup \powerset B$. So $x \in \powerset A$ or $x \in \powerset B$. Suppose $x \in \powerset A$. Since $A \subseteq A \cup B$, $x \in \powerset (A \cup B)$. The case $x \in \powerset B$ is similar.
\end{proof}

\begin{exthm}[7b]
  If $A \subseteq B$, then $\powerset A \cup \powerset B = \powerset(A \cup B)$.
\end{exthm}
\begin{proof}
  Since $A \subseteq B$, $\powerset A \subseteq \powerset B$, and so $\powerset A \cup \powerset B = \powerset B = \powerset (A \cup B)$.
\end{proof}

\begin{exthm}[8]
  There is no set to which every singleton belongs.
\end{exthm}
\begin{proof}
  Let $A$ be a set to which every singleton belongs. Let $b$ be an arbitrary set. Then $b \in \cup A$ as $\{b\}$ is a singleton and so is in $A$. Hence $\cup A$ contains every set.
\end{proof}

\begin{eg}[9]
  Let $a = \{1,2\}$ and $B = \{\{1,2\}\}$.
\end{eg}

\begin{exthm}[10]
  If $a \in B$, then $\powerset a \in \powerset \powerset \cup B$.
\end{exthm}
\begin{proof}
  This is equivalent to showing $\powerset a \subseteq \powerset \cup B$. Let $x \in \powerset a$, so $x \subseteq a$. Since $a \in B$, $a \subseteq \cup B$ (by Exercises~\ref{memberUnion}). So $x \subseteq \cup B$ or $x \in \powerset \cup B$.
\end{proof}

\subsubsection{Algebra of Sets}
\begin{exthm}[13]
  If $A \subseteq B$, then $C - B \subseteq C - A$.
\end{exthm}
\begin{proof}
  Let $x \in C - B$. Then $x \in C$ but $x \not\in B$. Since $A \subseteq B$, if $x \not\in B$, then $x \not\in A$. So $x \in C - A$.
\end{proof}

\begin{eg}[14]
  Let $A = \aSet{1,2}$ and $B = C = \aSet{2}$.
\end{eg}

\begin{exthm}[15a]
  $A \cap (B + C) = (A \cap B) + (A \cap C)$.
\end{exthm}
\begin{proof}
  Let $x \in A \cap (B + C)$. We consider two cases.
  \begin{itemize}
    \item $x \in A \cap B$ but $x \not\in C$. Then $x \not\in A \cap C$, so $x \in (A \cap B) + (A \cap C)$.
    \item The case $x \in A \cap C$ but $x \not\in B$ is similar.
  \end{itemize}
\end{proof}

\begin{exthm}[15b]
  $A + (B + C) = (A + B) + C$
\end{exthm}
\begin{proof}
  TODO
\end{proof}

\begin{exthm}[17]
  The following conditions are equivalent:
  \begin{multicols}{2}
    \begin{enumerate}[(a)]
      \item $A \subseteq B$
      \item $A - B = \varnothing$
      \item $A \cup B = B$
      \item $A \cap B = A$
    \end{enumerate}
  \end{multicols}
\end{exthm}
\begin{proof}
  We proceed by showing each condition implies each other.
  \begin{itemize}
    \item (b) $\Rightarrow$ (a)

      Suppose $A - B = \varnothing$. Then there does not exist a $x$ such that $x \in A$ and $x \not\in B$. Or for all $x$, $x \not\in A$ or $x \in B$, which means for all $x$, $x \in A$ implies $x \in B$. Hence $A \subseteq B$.
    \item (a) $\Rightarrow$ (c)

      Let $x \in A \cup B$, since $A \subseteq B$, $x \in B$.

    \item (c) $\Rightarrow$ (d)

      Suppose $A \cup B = B$, then $A \cap B = A \cap (A \cup B) = A$.

    \item (d) $\Rightarrow$ (b)

      Suppose $A \cap B = A$, then
      \[
        A - B = (A \cap B) - B = \Set*{x \st x \in A \cap B\ \text{and}\ x \not\in B} = \varnothing.
      \]

  \end{itemize}
\end{proof}

\begin{eg}[18]
  Assume that $A$ and $B$ are subsets of $S$. Then there are 15 different sets that can be made from these three by use of the binary operations $\cup$, $\cap$, and $-$.

  This can be seen by splitting the sets in 4 different sets, $S - (A \cup B)$, $A - B$, $B - A$, and $A \cap B$. For each of these sets, we can either include or exclude it. Excluding not selecting any, we have $2^4 - 1 = 15$ possible sets. These are,

  \begin{enumerate}
    \item $S - (A \cup B)$
    \item $A - B$
    \item $B - A$
    \item $A \cap B$
    \item $S - B$
    \item $S - A$
    \item $(A - B) \cup (B - A)$
    \item $(S - (A \cap B)) \cup (A \cap B)$ (complement of symmetric difference)
    \item $A$
    \item $B$
    \item $S - (A \cap B)$
    \item $A \cup (S - B)$
    \item $B \cup (S - A)$
    \item $A \cup B$
    \item $S$
  \end{enumerate}
\end{eg}

\begin{eg}[19]
  Since $\varnothing \in \powerset (A - B)$ and $\varnothing \not\in \powerset A - \powerset B$, $\powerset(A - B) \neq \powerset A - \powerset B$.
\end{eg}

\begin{exthm}[20]
  Let $A, B$ and $C$ be sets such that $A \cup B = A \cup C$ and $A \cap B = A \cap C$. Then $B = C$.
\end{exthm}
\begin{proof}
  Let $x \in B$. We consider two cases,
  \begin{itemize}
    \item Suppose $x \in A$. Then $x \in A \cap B = A \cap C$. So $x \in C$.
    \item Suppose $x \not\in A$. Then $x \not\in A \cap B = A \cap C$. But since $x \in A \cup C$, we have $x \in C - A$ and so $x \in C$.
  \end{itemize}
  This shows $B \subseteq C$. The same argument starting with $x \in C$ shows that $C \subseteq B$.
\end{proof}

\begin{exthm}[21]
  $\bigcup (A \cup B) = \bigcup A \cup \bigcup B$.
\end{exthm}
\begin{proof}
  \begin{align*}
    \bigcup (A \cup B) &= \Set*{x \st (\exists y \in A \cup B)\, x \in y}\\
                       &= \Set*{x \st (\exists y \in A)x \in y \text{ or } (\exists y \in B) x \in y}\\
                       &= \Set*{x \st (\exists y \in A)x \in y} \cup \Set*{x \st (\exists y \in B) x \in y}\\
                       &= \bigcup A \cup \bigcup B.
  \end{align*}
\end{proof}

\begin{exthm}[23]
  If $\mathscr{B}$ is nonempty, then $A \cup \bigcup \mathscr{B} = \bigcap \Set*{A \cup X \st X \in \mathscr{B}}$.
\end{exthm}
\begin{proof}
  \begin{align*}
    x \in A \cup \bigcap \mathscr{B} &\iff x \in A \text{ or } (\forall X \in \mathscr{B}) t \in X\\
                                     &\iff (\forall X \in \mathscr{B}) (t \in A \text{ or } t \in X)\\
                                     &\iff t \in \bigcap\Set*{A \cup X \st t \in \mathscr{B}}.
  \end{align*}
\end{proof}

\begin{exthm}[24a]
  If $A$ is nonempty, then $\powerset \bigcap A = \bigcap \Set{\powerset X \st X \in A}$.
\end{exthm}
\begin{proof}
  \begin{align*}
    t \in \powerset \bigcap \mathscr{A} &\iff t \subseteq \bigcap \mathscr{A}\\
                                        &\iff (\forall X \in \mathscr{A}) (t \subseteq X)\\
                                        &\iff (\forall X \in \mathscr{A}) (t \in \powerset X)\\
                                        &\iff t \in \bigcap \Set{\powerset X | X \in \mathscr{A}}.
  \end{align*}
\end{proof}

\begin{exthm}[24b]
  $\bigcup \Set{\powerset X \st X \in \mathscr{A}} \subseteq \powerset \bigcup \mathscr{A}$.
\end{exthm}
\begin{proof}
  Let $t \in \bigcup \Set*{\powerset X \st X \in \mathscr{A}}$. Then there is a $X \in \mathscr{A}$ such that $t \in \powerset X$. Since $t \in \powerset X$ then $t \subseteq X$, and since $X \in \mathscr{A}$, $X \subseteq \bigcup \mathscr{A}$. Putting these together, we have $t \subseteq X \subseteq \bigcup \mathscr{A}$. So $t \in \powerset \bigcup \mathscr{A}$.
\end{proof}
\begin{remark}
  Equality never holds.
\end{remark}

\begin{exthm}[25]
  $A \cup \bigcup \mathscr{B} = \bigcup\Set*{A \cup X \st X \in \mathscr{B}}$.
\end{exthm}
\begin{proof}
  TODO
\end{proof}

\subsubsection{Review Exercises}
\begin{exthm}[27]
  $\bigcap A \cap \bigcap B \neq \bigcap(A \cap B)$.
\end{exthm}
\begin{proof}
  Let $A = \{\{1\}, \{2\}\}$, and $B = \{1\}$.
\end{proof}

\subsection{Relations and Functions}
\begin{eg}[1]
  \begin{align*}
    \ang{\varnothing, \varnothing, \aSet{\varnothing}}^* &=
    \aSet{\aSet{\varnothing}, \aSet{\varnothing, \varnothing}, \aSet{\varnothing, \varnothing, \aSet{\varnothing}}}\\
                                                         &=  \aSet{\aSet{\varnothing}, \aSet{\varnothing, \aSet{\varnothing}}, \aSet{\varnothing, \aSet{\varnothing}, \varnothing}}\\
                                                         &= \ang{\varnothing, \aSet{\varnothing}, \varnothing}^*
  \end{align*}
\end{eg}
\begin{exthm}[4]
  There is no set to which every ordered pair belongs.
\end{exthm}
\begin{proof}
  An ordered pair is a singleton, which we have shown there is no such set.
\end{proof}
\begin{exthm}[5a]
  Assume $A$ and $B$ are given sets. There exists a set $C$ such that for any $y$,
  \[
    y \in C \iff y = \{x\} \times B\ \text{for some $x$ in $A$}.
  \]
\end{exthm}
\begin{proof}
  By a subset axiom,
  \[
    C = \Set*{y \in \powerset (A \times B) \st y = \{x\} \times B\ \text{for some $x$ in $A$}}.
  \]
\end{proof}
\begin{exthm}[5b]
  With $A$, $B$, and $C$ being defined in the previous exercise, $A \times B = \bigcup C$.
\end{exthm}
\begin{proof}
  \begin{align*}
    \bigcup C &= \Set*{y \st y \in X\ \text{for some $X$ in $C$}}\\
              &= \Set*{y \st y \in \{x\} \times B\ \text{for some $x$ in $A$}}\\
              &= A \times B.
  \end{align*}
\end{proof}

\subsubsection{Relations}

\begin{exthm}[6]
  A set $A$ is a relation if and only if $A \subseteq \dom{A} \times \ran{A}$.
  \begin{itemize}
    \item ($\Rightarrow$) Suppose $A$ is a relation. Let $\ang{x,y} \in A$. Then $x \in \dom{A}$, as $\ang{x,y} \in A$, and similarly, $y \in \ran{A}$. So $\ang{x,y} \in \dom{A} \times \ran{A}$.
    \item ($\Leftarrow$) Suppose $A \subseteq \dom{A} \times \ran{A}$. Suppose $t \in A$ an $t$ is not an ordered pair. But $t \in A$ so $t \in \dom{A} \times \ran{A}$. Therefore $t$ is an element of a set of ordered pairs.
  \end{itemize}
\end{exthm}

\begin{exthm}[7]
  If $R$ is a relation, then $\fld{R} = \bigcup\bigcup R$.
\end{exthm}
\begin{proof}
  Let $x \in \dom{R}$. Since $\dom{R} = \Set*{x \in \bigcup\bigcup R \st \exists y \ang{x,y} \in R}$, $x \in \bigcup\bigcup R$.

  Let $x \in \bigcup\bigcup R$. Since $R$ a relation, $R = \Set{\ang{x, y} \st \ang{x,y} \in R}$. So $\bigcup \bigcup R = \Set*{u \st \exists \ang{x,y} \in R \text{ and } u = x \text{ or } u = y}$. Hence $x \in \dom{R} \cup \ran{R}$.
\end{proof}

\begin{exthm}[8]
  For any set $\mathscr{A}$,
  \begin{align*}
    \dom{\bigcup \mathscr{A}} &= \bigcup \Set*{\dom{R} \st R \in \mathscr{A}}\\
    \ran{\bigcup \mathscr{A}} &= \bigcup \Set{\ran{R} \st R \in \mathscr{A}}
  \end{align*}
\end{exthm}
\begin{proof}
  We have,
  \begin{align*}
    \dom{\bigcup \mathscr{A}} &= \Set{x \st \exists y\, \ang{x, y} \in \bigcup \mathscr{A}}\\
                              &= \Set*{x \st \exists y\, \exists R \in \mathscr{A}\ \ang{x,y} \in R}\\
                              &= \Set*{x \st (\exists R \in \mathscr{A})\ x \in \dom{R}}\\
                              &= \bigcup \Set*{\dom{R} \st R \in \mathscr{A}}
  \end{align*}
  A similar argument holds for $\ran \bigcup \mathscr{A}$.
\end{proof}

\begin{remark}[9]
  If union were replaced with intersection in the previous question we could not interchange the $\exists$ and $\forall$.
\end{remark}

\subsubsection{n-ary Relations}
\subsubsection{Functions}

\begin{exthm}[12]
  Suppose $f$ and $g$ are functions. Then
  \[
    f \subseteq g \iff \dom{f} \subseteq g \text{ and  } (\forall x \in \dom{f}) f(x) = g(x).
  \]
\end{exthm}
\begin{proof}
  Suppose $f \subseteq g$ and let $x \in \dom{f}$. Then there exists $y$ such that $\ang{x, y} \in f \subseteq g$, so $x \in \dom{g}$. Since $xfy$ and $xgy$, we have $f(x) = g(x)$. And so on.
\end{proof}

\begin{exthm}[13]
  If $f$ and $g$ are functions with $f \subseteq g$ and $dom g \subseteq \dom{f}$, then $f = g$.
\end{exthm}
\begin{proof}
  This follows from the previous exercise.
\end{proof}

\begin{exthm}[14a]
  Let $f$ and $g$ be functions. Then $f \cap g$ is a function.
\end{exthm}
\begin{proof}
  Let $x(f\cap g)y$ and $x(f \cap g)y'$. Then since $f$ and $g$ are functions, it follows $y = y'$.
\end{proof}

\begin{exthm}[14b]
  If $f$ and $g$ are functions, then $f \cup g$ is a function iff $f(x) = g(x)$ for every $x$ in $\dom{f} \cap \dom{g}$.
\end{exthm}
\begin{proof}
  Let $x \in \dom{f} \cup \dom{g}$ such that $f(x) \neq g(x)$. Then $\ang{x, f(x)} \in f$ and $\ang{x, g(x)} \in g$. So $\ang{x, f(x)} \in f \cup g$, and $\ang{x, g(x)} \in f \cup g$, but $f(x) \neq g(x)$, so $f \cup g$ is not a function.

  Similar logic shows that if $f(x) = g(x)$ in the shared domain, then $f \cup g$ is function.
\end{proof}

\begin{exthm}[15]
  Let $\mathscr{A}$ be a set of functions such that for any $f$ and $g$ in $\mathscr{A}$, either $f \subseteq g$ or $g \subseteq f$. Then $\bigcup \mathscr{A}$ is a function.
\end{exthm}
\begin{proof}
  Suppose $x \bigcup \mathscr{A} y$ and $x \bigcup \mathscr{A} y'$, then $\ang{x,y} \in f$, for some $f \in \mathscr{A}$, and $\ang{x, y'} \in g$, for some $g \in \mathscr{A}$. Without loss of generality, suppose $f \subseteq g$, then $\ang{x, y} \in g$. Since $g$ is a function, it follows $y = y'$. Hence, $\bigcup \mathscr{A}$ is a function.
\end{proof}

\begin{exthm}[22(b)]
  For any sets, $(F \circ G)[A] = F[G[A]]$.
\end{exthm}
\begin{proof}
  \begin{align*}
    y \in (F \circ G)[A] &\iff \exists x \in A\, x (F \circ G)y\\
                         &\iff \exists x \in A\, \exists t (xGt \land tFy)\\
                         &\iff \exists t \in G[A]\ tFy\\
                         &\iff y \in F[G[A]].
  \end{align*}
\end{proof}

\begin{exthm}[23]
  Let $I_A$ be the identity function on the set $A$. For any sets $B$ and $C$, $B \circ I_A = B \restriction A$ and $I_A[C] = A \cap C$.
\end{exthm}
\begin{proof}
  For the first,
  \begin{align*}
    B \circ I_A &= \Set{\ang{x,y} \st \exists t (xI_A t \land tBy)}\\
                &= \Set{\ang{x,y} \st \exists t (t \in A \land tBy)}\\
                &= B \restriction A.
  \end{align*}

For the second,

\begin{align*}
  I_A [C] &= \Set{y \st \exists x \in C\ \ang{x, y} \in I_A}\\
          &= \Set{y \st \exists y (y \in C \land y \in I_A)}\\
          &= A \cap C.
\end{align*}
\end{proof}

\begin{exthm}[24]
  For a function $F$, $F^{-1}[A] = \Set{x \in dom{F} \st F(x) \in A}$.
\end{exthm}
\begin{proof}
  Since $F^{-1} = \Set{\ang{y,x} \st \ang{x, y} \in F}$, we have
  \begin{align*}
    F^{-1}[A] &= \ran{(F^{-1} \restriction A)}\\
              &= \Set{x \st \exists y \in A\ y F^{-1} x}\\
              &= \Set{x \st \exists y \in A\ x F y}\\
              &= \Set{x \in \dom{F} \st F(x) \in A}.
  \end{align*}
\end{proof}

\begin{exthm}[25]
  Suppose $G$ is a function, then $G \circ G^{-1} = I_{\ran{G}}$.
\end{exthm}
\begin{proof}
  We have,
  \begin{align*}
    G \circ G^{-1} &= \Set{\ang{x,y} \st \exists t\ x G^{-1} t \land tGy}\\
                   &= \Set{\ang{x,y} \st \exists t\ tGx \land tGy}
  \end{align*}
  But since $G$ is a function, $x = y$, so
  \begin{align*}
    G \circ G^{-1} &= \Set{\ang{x,x} \st \exists t\ tGx}\\
                   &= \Set{\ang{x,x} \st x \in \ran{G}}.
  \end{align*}
\end{proof}

\begin{exthm}[26]
  Let $F$ be a set. Then $F[\bigcup \mathscr{A}] = \bigcup \Set*{F[A] \st A \in \mathscr{A}}$.
\end{exthm}
\begin{proof}
  We have,
  \begin{align*}
    y \in F[\bigcup \mathscr{A}] &\iff (\exists x \in \bigcup \mathscr{A}) xFy\\
                                 &\iff (\exists A \in \mathscr{A} \exists x \in A) xFy\\
                                 &\iff (\exists A \in \mathscr{A}) y \in F[A]\\
                                 &\iff y \in \bigcup \Set*{F[A] \st A \in \mathscr{A}}.
  \end{align*}
\end{proof}

\begin{exthm}[27]
  For any sets $F$ and $G$, $\dom{F \circ G} = G^{-1}[\dom{F}]$.
\end{exthm}
\begin{proof}
  We have,
  \begin{align*}
    \dom{F \circ G} &= \Set*{u \st \exists y \ang{u, v} \in F \circ G}\\
                    &= \Set*{u \st (\exists v \exists t) uGt \text{ and } tFv}\\
                    &= \Set*{u \st \exists t \in \dom{F}\ uGt}\\
                    &= \Set*{u \st (\exists t \in \dom{F})\ tG^{-1}u}\\
                    &= \ran{(G^{-1} \restriction \dom F)}\\
                    &= G^{-1}[\dom{F}].
  \end{align*}
\end{proof}

\begin{exthm}[28]
  If $f$ is a one-to-one function from $A$ into $B$, and that $G$ is the function with $\dom{G} = \powerset{A}$ defined by $G(X) = f[X]$, then $G$ maps $\powerset{A}$ one-to-one into $\powerset{B}$.
\end{exthm}
\begin{proof}
  Let $X$ and $Y$ be arbitrary elements of $\powerset{A}$ and that $G(X) = G(Y)$. Suppose $x \in X$. Then $f(x) \in f[X] = G(X) = G(Y) = f[Y]$. So $x \in Y$. Therefore, $X \subseteq Y$. Similar reasoning shows that $Y \subseteq X$. And so, $X = Y$.
\end{proof}

\begin{exthm}[29]
  Assume that $f: A \rightarrow B$ and define $G: B \rightarrow \powerset{A}$ by $G(b) = \Set*{x \in A \st f(x) = b}$. Then if $f$ maps $A$ onto $B$, then $G$ is one-to-one.
\end{exthm}
\begin{proof}
  TODO
\end{proof}

\subsubsection{Infinite Cartesian Products}

\begin{exthm}[31]
  The following are equivalent:
  \begin{itemize}
    \item For any relation $R$ there is a function $H \subseteq R$ with $\dom{H} = \dom{R}$.
    \item For any set $I$ and any function $H$ with domain $I$, if $H(i) \neq \varnothing$ for all $i$ in $I$, then $\bigtimes_{i \in I} H(i) \neq \varnothing$.
\end{itemize}
\end{exthm}
\begin{proof}
  \begin{itemize}
    \item ($\Rightarrow$) Let $I$ be a set and $H$ be a function with $\dom{H} = I$. Suppose
      TODO
  \end{itemize}
\end{proof}

\subsubsection{Equivalence Relations}

\begin{exthm}[34a]
  Suppose $\mathscr{A}$ is a nonempty set, every member is a transitive relation. Then $\bigcap \mathscr{A}$ is a transitive relation.
\end{exthm}
\begin{proof}
  Let $\ang{x,y}, \ang{y,z} \in \bigcap \mathscr{A}$. There for all $R \in \mathscr{A}$, $\ang{x,y} \in R$, and $\ang{y,z} \in R$. Since $R$ is transitive, $\ang{x, z} \in R$.
\end{proof}

\begin{exthm}[34a]
  Suppose $\mathscr{A}$ is a nonempty set, every member is a transitive relation. Then $\bigcup \mathscr{A}$ is not a transitive relation.
\end{exthm}
\begin{proof}
  Let $\mathscr{A} = \{\{1, 2\}, \{2, 3\}\}$.
\end{proof}

\begin{exthm}[35]
  For any $R$ and $x$, $[x]_R = R[\{x\}]$.
\end{exthm}
\begin{proof}
  \begin{align*}
    R[\{x\}] &= \ran(R \restriction \{x\})\\
             &= \Set*{y \st \exists u \in \{x\}, uRy}\\
             &= \Set*{y \st xRy}\\
             &= [x]_R.
  \end{align*}
\end{proof}

\subsection{Natural numbers}
\subsubsection{Peano's Postulates}
\begin{exthm}[2]
  If $a$ is a transitive set, then $a^+$ is a transitive set.
\end{exthm}
\begin{proof}
  We will show that $\bigcup (a^+) \subseteq a^+$, showing $a^+$ is a transitive set. Since $\bigcup (a^+) = a \subseteq a^+$.
\end{proof}

\begin{exthm}[3a]
  If $a$ is a transitive set, then $\powerset a$ is a transitive set.
\end{exthm}
\begin{proof}
  Firstly, $\bigcup \powerset(a) = a$. Then since $a$ is transitive, $a \subseteq \powerset(a)$, so $\bigcup \powerset(a) \subseteq \powerset(a)$.
\end{proof}

\begin{exthm}[3b]
  If $\powerset a$ is a transitive set, then $a$ is a transitive set.
\end{exthm}
\begin{proof}
  Let $x \in a$. Then for some $y \in \powerset a$, $x \in y \in \powerset a$. Since $\powerset a$ is a transitive set, $x \in \powerset a$, or $x \subseteq a$. Hence, $a$ is a transitive set.
\end{proof}

\begin{exthm}
  If $a$ is a transitive set, then $\bigcup a$ is also a transitive set.
\end{exthm}
\begin{proof}
  Let $x \in y \in \bigcup a$. Then there is a $Y \in a$ such that $y \in Y \in a$. Since $a$ is a transitive set, $y \in a$. So $x \in y \in a$. Therefore $x \in \bigcup a$.
\end{proof}

\begin{exthm}[5a]
  If every member of $\mathscr{A}$ is a transitive set, then $\bigcup \mathscr{A}$ is a transitive set.
\end{exthm}
\begin{proof}
  Let $a \in \bigcup \mathscr{A}$. Then there is an $A \in \mathscr{A}$ such that $a \in A$. Since $A$ is transitive, $a \subseteq A$, and so $a \subseteq \bigcup \mathscr{A}$.
\end{proof}

\begin{exthm}[6]
  If $\bigcup(a^+) = a$, then $a$ is a transitive set.
\end{exthm}
\begin{proof}
  Suppose $a$ is a transitive set. Let $x \in y \in \bigcup(a^+)$. Then there is a $Y \in a^+$, such that $x \in Y$.

  So $x \in Y \in a^+ = a \cup \{a\}$. If $x \in Y \in a$, then since $a$ is a transitive set, $x \in a$.

  If $x \in Y \in \{a\}$, then $x \in Y = a$. Hence $x \in a$.
\end{proof}

\begin{exthm}[7]
  Let $A$ be a set, $a \in A$, and $F: A \mapsto A$. Then there exists a unique function $h: \omega \mapsto A$ such that
\begin{align}
  h(0) &= a,\\
  h(n^+) &= F(h(n)), \ \text{for every $n$ in $\omega$.}
  \end{align}
\end{exthm}
\begin{proof}
  This is a proof of the uniqueness. Let $h_1$ and $h_2$ both satisfy the conclusion of the theorem. Let $S = \Set*{n \in \omega \st h_1(n) = h_2(n)}$. We will show $S$ is inductive.

  $0 \in S$, as both $h_1$ and $h_2$ are acceptable, so $h_1(0) = a = h_2(0)$.

  Suppose $k \in S$. Then since $h_1$ is acceptable,
  \begin{align*}
    h_1(k^+) &= F(h_1(k))\\
             &= F(h_2(k))\\
             &= h_2(k^+).
  \end{align*}
  So $k^+ \in S$. Hence, $S$ is inductive. Therefore, $h$ is unique.
\end{proof}

\begin{exthm}[8]
  Let $f$ be a one-to-one function from $A$ into $A$, and assume that $c \in A - \ran{f}$. Define $h: \omega \mapsto A$ by recursion:
  \begin{align*}
    h(0) &= c\\
    h(n^+) &= f(h(n)).
  \end{align*}
  Then $h$ is one-to-one.
\end{exthm}
\begin{proof}
  Let $T = \Set*{n \in \omega \st h(m) = h(n) \implies m = n}$.

  First we will show that $0 \in T$. Let $m \in \omega$, and $m \neq 0$, then $m = p^+$, for some $p \in \omega$. Then
\begin{align*}
  h(p^+) &= f(h(p))\\
         &\in \ran{f}.
\end{align*}
So $h(p^+) = h(m) \neq c$. So $h(m) \neq h(0)$ when $m \neq 0$.

The proof that $k \in T$ implies $k^+ \in T$ is similar to Theorem 4H.
\end{proof}
\end{document}
